% Options for packages loaded elsewhere
\PassOptionsToPackage{unicode}{hyperref}
\PassOptionsToPackage{hyphens}{url}
%
\documentclass[
]{book}
\usepackage{lmodern}
\usepackage{amssymb,amsmath}
\usepackage{ifxetex,ifluatex}
\ifnum 0\ifxetex 1\fi\ifluatex 1\fi=0 % if pdftex
  \usepackage[T1]{fontenc}
  \usepackage[utf8]{inputenc}
  \usepackage{textcomp} % provide euro and other symbols
\else % if luatex or xetex
  \usepackage{unicode-math}
  \defaultfontfeatures{Scale=MatchLowercase}
  \defaultfontfeatures[\rmfamily]{Ligatures=TeX,Scale=1}
\fi
% Use upquote if available, for straight quotes in verbatim environments
\IfFileExists{upquote.sty}{\usepackage{upquote}}{}
\IfFileExists{microtype.sty}{% use microtype if available
  \usepackage[]{microtype}
  \UseMicrotypeSet[protrusion]{basicmath} % disable protrusion for tt fonts
}{}
\makeatletter
\@ifundefined{KOMAClassName}{% if non-KOMA class
  \IfFileExists{parskip.sty}{%
    \usepackage{parskip}
  }{% else
    \setlength{\parindent}{0pt}
    \setlength{\parskip}{6pt plus 2pt minus 1pt}}
}{% if KOMA class
  \KOMAoptions{parskip=half}}
\makeatother
\usepackage{xcolor}
\IfFileExists{xurl.sty}{\usepackage{xurl}}{} % add URL line breaks if available
\IfFileExists{bookmark.sty}{\usepackage{bookmark}}{\usepackage{hyperref}}
\hypersetup{
  pdftitle={金融数学},
  pdfauthor={Financial Mathematics},
  hidelinks,
  pdfcreator={LaTeX via pandoc}}
\urlstyle{same} % disable monospaced font for URLs
\usepackage{longtable,booktabs}
% Correct order of tables after \paragraph or \subparagraph
\usepackage{etoolbox}
\makeatletter
\patchcmd\longtable{\par}{\if@noskipsec\mbox{}\fi\par}{}{}
\makeatother
% Allow footnotes in longtable head/foot
\IfFileExists{footnotehyper.sty}{\usepackage{footnotehyper}}{\usepackage{footnote}}
\makesavenoteenv{longtable}
\usepackage{graphicx,grffile}
\makeatletter
\def\maxwidth{\ifdim\Gin@nat@width>\linewidth\linewidth\else\Gin@nat@width\fi}
\def\maxheight{\ifdim\Gin@nat@height>\textheight\textheight\else\Gin@nat@height\fi}
\makeatother
% Scale images if necessary, so that they will not overflow the page
% margins by default, and it is still possible to overwrite the defaults
% using explicit options in \includegraphics[width, height, ...]{}
\setkeys{Gin}{width=\maxwidth,height=\maxheight,keepaspectratio}
% Set default figure placement to htbp
\makeatletter
\def\fps@figure{htbp}
\makeatother
\setlength{\emergencystretch}{3em} % prevent overfull lines
\providecommand{\tightlist}{%
  \setlength{\itemsep}{0pt}\setlength{\parskip}{0pt}}
\setcounter{secnumdepth}{5}
\usepackage{booktabs}
\usepackage{ctex}
\usepackage{amsthm}
\makeatletter
\def\thm@space@setup{%
  \thm@preskip=8pt plus 2pt minus 4pt
  \thm@postskip=\thm@preskip
}
\makeatother
\usepackage[]{natbib}
\bibliographystyle{apalike}

\title{金融数学}
\author{Financial Mathematics}
\date{2020-09-17 16:50:46}

\begin{document}
\maketitle

{
\setcounter{tocdepth}{1}
\tableofcontents
}
\hypertarget{ux6b22ux8fce}{%
\chapter*{欢迎}\label{ux6b22ux8fce}}
\addcontentsline{toc}{chapter}{欢迎}

在这里,我们同步课堂,总结每章的\textbf{重点、难点},并发布\textbf{课后作业}。课后作业需在下次上课前交到讲台上。

我们这里主要以英文表述,有以下两个原因

\begin{enumerate}
\def\labelenumi{\arabic{enumi}.}
\item
  方便大家准备SOA/CAS的 \href{https://www.soa.org/education/exam-req/edu-exam-fm-detail/}{Exam FM: Financial Mathematics}考试;
\item
  方便大家阅读相关英文文献。
\end{enumerate}

此网站由授课老师高光远、助教程轶鹏、助教胡夏新管理,欢迎大家反馈意见到助教、微信群、或邮箱\href{mailto:guangyuan.gao@ruc.edu.cn}{\nolinkurl{guangyuan.gao@ruc.edu.cn}}。

\hypertarget{ux7b54ux7591}{%
\section*{答疑}\label{ux7b54ux7591}}
\addcontentsline{toc}{section}{答疑}

我定期把同学们的普遍疑问在这里解答,欢迎提问!

\textbf{\(i\) 和 \(d\) 的关系} (2020/09/16)

很多同学问课件上的这道题目。

问题:已知年实际利率为5\%。回答下述问题:

(1)100万元贷款在年末的利息是多少?\(100\times5%
\)

(2)如果在贷款起始日收取利息,应该收取多少利息?\(100\times i/(1+i)=100\times d\)

(3)年实际贴现率是多少?\(d=i/(1+i)\)

\(i\) 和 \(d\) 的区别可以理解为 \(i\) 是在\textbf{期末}付,\(d\) 是在\textbf{期初}付。\(d=i\times v\),即期末 \(i\) 的\textbf{现值}是 \(d\)。

所以(1)是期末收的利息,(2)是期初收的利息。期初收的利息要比期末收的少,因为银行收到的这部分利息在这一年中还能产生利息,期初收的 \(d\) 到期末是 \(i\)。

贴现率 \(d\) 的另一种理解就是利息 \(i\) 的现值。

\textbf{计算器} (2020/09/10)

在课堂测验和期末考试,没有对计算器的严格要求,但至少需要科学计算器。大家不需要购买昂贵的可编程计算器,在这门课中,体现不出可编程计算器的优势。

建议的计算器是SOA/CAS要求的\href{https://www.soa.org/education/exam-req/exam-day-info/edu-id-calculators/}{计算器}。

\textbf{最终成绩} (2020/09/10)

\begin{enumerate}
\def\labelenumi{\arabic{enumi}.}
\item
  平时成绩占40\%,期末成绩占60\%。
\item
  平时成绩主要根据课堂点名、课外作业的完成态度、随堂测试的准确度评定。
\end{enumerate}

\hypertarget{interest-rate}{%
\chapter{Interest rate}\label{interest-rate}}

\hypertarget{key-concepts}{%
\section{Key concepts}\label{key-concepts}}

\hypertarget{functions}{%
\subsection*{Functions}\label{functions}}
\addcontentsline{toc}{subsection}{Functions}

\begin{itemize}
\item
  Accumulation function \[a(t)\]
\item
  Discount function \[a^{-1}(t)\]
\end{itemize}

\hypertarget{interest-rate-1}{%
\subsection*{Interest rate}\label{interest-rate-1}}
\addcontentsline{toc}{subsection}{Interest rate}

\begin{itemize}
\item
  Effective rate of interest/discount \[i,d\]
\item
  Simple interest \[a(t)=1+it\]
\item
  Compound interest \[a(t)=(1+i)^t\]
\item
  Discount factor \[v=(1+i)^{-1}\]
\item
  Accumulation factor of \(t\) years \[(1+i)^t\]
\item
  Discount factor of \(t\) years \[(1+i)^{-t}\]
\item
  Nominal rate of interest/discount \[i^{(m)},d^{(m)}\]
\item
  Force of interest \[\delta\]
\end{itemize}

\hypertarget{values}{%
\subsection*{Values}\label{values}}
\addcontentsline{toc}{subsection}{Values}

\begin{itemize}
\item
  Accumulated value (future value)
\item
  Present value
\end{itemize}

\hypertarget{key-equations}{%
\section{Key equations}\label{key-equations}}

\hypertarget{accumulation-and-discount}{%
\subsection*{Accumulation and discount}\label{accumulation-and-discount}}
\addcontentsline{toc}{subsection}{Accumulation and discount}

\[a(t)=(1+i)^t=(1-d)^{-t}\]

\[a^{-1}(t)=(1+i)^{-t}=(1-d)^t=v^t\]

\hypertarget{effective-interest-rate-and-discount-rate}{%
\subsection*{Effective interest rate and discount rate}\label{effective-interest-rate-and-discount-rate}}
\addcontentsline{toc}{subsection}{Effective interest rate and discount rate}

\[i=\frac{d}{1-d}\]

\[d=\frac{i}{1+i}\]

\[d=iv\]

\[v=1-d\]

\[i-d=id\]

\hypertarget{nominal-interest-rate-and-effective-interest-rate}{%
\subsection*{Nominal interest rate and effective interest rate}\label{nominal-interest-rate-and-effective-interest-rate}}
\addcontentsline{toc}{subsection}{Nominal interest rate and effective interest rate}

\[1+i=\left(1+\frac{i^{(m)}}{m}\right)^m\]
\[1-d=\left(1-\frac{d^{(m)}}{m}\right)^m\]
\[d^{(m)}=i^{(m)}\times\left(1+\frac{i^{(m)}}{m}\right)^{-1}\]

\hypertarget{force-of-interest}{%
\subsection*{Force of interest}\label{force-of-interest}}
\addcontentsline{toc}{subsection}{Force of interest}

\[\delta(t)=\frac{a'(t)}{a(t)}\]

\[a(t)=e^{\int_0^t\delta(s)ds}\]

\[\delta=\ln(1+i)\]
\[\delta=\lim_{m\rightarrow\infty} i^{(m)}=\lim_{m\rightarrow\infty} d^{(m)}=\ln(1+i)\]

\[d\le d^{(2)}\le d^{(3)}\le\cdots\le \delta\le\cdots\le i^{(3)}\le i^{(2)}\le i\]

\hypertarget{level-annuity}{%
\chapter{Level annuity}\label{level-annuity}}

\hypertarget{key-concepts}{%
\section{Key concepts}\label{key-concepts}}

\hypertarget{annuity-immediate}{%
\subsection*{Annuity immediate}\label{annuity-immediate}}
\addcontentsline{toc}{subsection}{Annuity immediate}

\hypertarget{annuity-due}{%
\subsection*{Annuity due}\label{annuity-due}}
\addcontentsline{toc}{subsection}{Annuity due}

\hypertarget{deffered-annuity}{%
\subsection*{Deffered annuity}\label{deffered-annuity}}
\addcontentsline{toc}{subsection}{Deffered annuity}

\hypertarget{perpetuity}{%
\subsection*{Perpetuity}\label{perpetuity}}
\addcontentsline{toc}{subsection}{Perpetuity}

\hypertarget{m-thly-payable-annuity}{%
\subsection*{\texorpdfstring{\(m\)-thly payable annuity}{m-thly payable annuity}}\label{m-thly-payable-annuity}}
\addcontentsline{toc}{subsection}{\(m\)-thly payable annuity}

\hypertarget{continuous-payable-annuity}{%
\subsection*{Continuous payable annuity}\label{continuous-payable-annuity}}
\addcontentsline{toc}{subsection}{Continuous payable annuity}

\hypertarget{key-equations}{%
\section{Key equations}\label{key-equations}}

\hypertarget{homework}{%
\chapter*{Homework}\label{homework}}
\addcontentsline{toc}{chapter}{Homework}

\hypertarget{week-2}{%
\section*{Week 2}\label{week-2}}
\addcontentsline{toc}{section}{Week 2}

\hypertarget{problem-1}{%
\subsection*{Problem 1}\label{problem-1}}
\addcontentsline{toc}{subsection}{Problem 1}

\emph{SOA 5/98 \#2}

John invests \(1000\) in a fund which earns interest during the first year at a nominal rate of \(K\) convertible quarterly. During the 2nd year the fund earns interest at a nominal discount rate of \(K\) convertible quarterly. At the end of the 2nd year,the fund has accumulated to \(1173.54\).

Calculate \(K\).

\hypertarget{problem-2}{%
\subsection*{Problem 2}\label{problem-2}}
\addcontentsline{toc}{subsection}{Problem 2}

\emph{SOA 5/89 \#4}

Two funds,\(X\) and \(Y\),start with the same amount.You are given:

\begin{enumerate}
\def\labelenumi{\arabic{enumi}.}
\item
  Fund \(X\) accumulates at a force of interest of 5\%.
\item
  Fund \(Y\) accumulates at a rate of interest \(j\), compounded semiannually.
\item
  At the end of eight years, Fund \(X\) is \(1.05\) times as large as Fund \(Y\).
\end{enumerate}

Calculate \(j\).

\hypertarget{problem-3}{%
\subsection*{Problem 3}\label{problem-3}}
\addcontentsline{toc}{subsection}{Problem 3}

\emph{SOA 11/89 \#2}

Fund \(X\) starts with \(1,000\) and accumulates with a force of interest \[\delta_{t}=\frac{1}{15-t} \text{ for } 0 \le t< 15.\]

Fund Y starts with \(1,000\) and accumulates with an interest rate of 8\% per annum compounded semiannually for the first three years and an effective interest rate of \(i\) per annum thereafter.

Fund \(X\) equals Fund \(Y\) at the end of four years.

Calculate \(i\).

\hypertarget{week-1}{%
\section*{Week 1}\label{week-1}}
\addcontentsline{toc}{section}{Week 1}

\hypertarget{problem-1-1}{%
\subsection*{Problem 1}\label{problem-1-1}}
\addcontentsline{toc}{subsection}{Problem 1}

John invests \(X\) in a fund growing in accordance with the accumulation function implied by the \emph{amount function}
\[A(t)=4t^2+8t+4.\]
Edna invests \(X\) in another fund growing in accordance with the accumulation function implied by the amount function \[A(t)=4t^2+2.\]
When does Edna's investment \emph{exceed} John's?

\hypertarget{problem-2-1}{%
\subsection*{Problem 2}\label{problem-2-1}}
\addcontentsline{toc}{subsection}{Problem 2}

What deposit made today will provide for a payment of \(\$1000\) in 1 year and \(\$2000\) in 3 years, if the effective rate of interest is \(7.5\%\)?

\hypertarget{problem-3-1}{%
\subsection*{Problem 3}\label{problem-3-1}}
\addcontentsline{toc}{subsection}{Problem 3}

Company \(X\) received the approval to start no more than two projects in the current calendar year.
Three different projects were recommended, each of which requires an investment of 800 to be made at the beginning of the year.

The cash flows for each of the three projects are shown in Table \ref{tab:week1}:

\begin{table}

\caption{\label{tab:week1}The cash flows of the three projects.}
\centering
\begin{tabular}[t]{r|r|r|r}
\hline
End of year & Project A & Project B & Project C\\
\hline
1 & 500 & 500 & 500\\
\hline
2 & 500 & 300 & 250\\
\hline
3 & -175 & -175 & -175\\
\hline
4 & 100 & 150 & 200\\
\hline
5 & 0 & 200 & 200\\
\hline
\end{tabular}
\end{table}

The company uses an annual effective interest rate of \(10\%\) to discount its cash flows.

Determine which combination of projects the company should select.

\hypertarget{solutions-to-homework}{%
\chapter*{Solutions to homework}\label{solutions-to-homework}}
\addcontentsline{toc}{chapter}{Solutions to homework}

\hypertarget{week-1}{%
\section*{Week 1}\label{week-1}}
\addcontentsline{toc}{section}{Week 1}

\hypertarget{problem-1}{%
\subsection*{Problem 1}\label{problem-1}}
\addcontentsline{toc}{subsection}{Problem 1}

To compare the two funds, we assume that equal investments of \(X\) are made at time 0.

John's \textbf{accumulation function} is \[t^2+2t+1\]

Edna's \textbf{accumulation function} is \[2t^{2}+1\]

To determine when Edna's investment exceeds John's, we set:

\[ X(2t^{2}+1)>X(t^{2}+2t+1)\]

which reduces to:

\[t^{2}-2t>0\] or \[t(t-2)>0\]

Thus, Edna's fund exceeds John's after 2 years.

\hypertarget{problem-2}{%
\subsection*{Problem 2}\label{problem-2}}
\addcontentsline{toc}{subsection}{Problem 2}

\[PV=1000v+2000v^{3}=2540.15 \]

since \[v=1.075^{-1}\]

\hypertarget{problem-3}{%
\subsection*{Problem 3}\label{problem-3}}
\addcontentsline{toc}{subsection}{Problem 3}

Discounting at \(10\%\), the net present values are \(4.59\),\(-2.36\) and \(-9.54\) for Projects A,B,and C respectively.

Take Project A as an example:

\[NPV=-800+500v+500v^{2}-175v^{3}+100v^{4}=4.59\]

since \[v=1.10^{-1}\]

Hence, only Project A should be funded.

  \bibliography{\_reference.bib}

\end{document}
